% vim: set textwidth=120:

% Example CV based on the 1.5-column-cv template. Main features:
% * uses the Roboto font family and IcoMoon icon set;
% * doesn't use colours, different font weights are used instead for styling;
% * because the CV fits on one page, header and footer is empty, since there isn't much useful info to put there;
% * includes a photo.
\documentclass[a4paper,10pt]{article}


% package imports
% ---------------

\usepackage[british]{babel} % for correct language and hyphenation and stuff
\usepackage{calc}           % for easier length calculations (infix notation)
\usepackage{enumitem}       % for configuring list environments
\usepackage{fancyhdr}       % for setting header and footer
\usepackage{fontspec}       % for fonts
\usepackage{geometry}       % for setting margins (\newgeometry)
\usepackage{graphicx}       % for pictures
\usepackage{microtype}      % for microtypography stuff
\usepackage{xcolor}         % for colours
\usepackage{hyperref}
\hypersetup{
    colorlinks=true,
    linkcolor=blue,
    urlcolor=black,
}


% margin and column widths
% ------------------------

% margins
\newgeometry{left=15mm,right=15mm,top=15mm,bottom=15mm}

% width of the gap between left and right column
\newlength{\cvcolumngapwidth}
\setlength{\cvcolumngapwidth}{3.5mm}

% left column width
\newlength{\cvleftcolumnwidth}
\setlength{\cvleftcolumnwidth}{44.5mm}

% right column width
\newlength{\cvrightcolumnwidth}
\setlength{\cvrightcolumnwidth}{\textwidth-\cvleftcolumnwidth-\cvcolumngapwidth}

% set paragraph indentation to 0, because it screws up the whole layout otherwise
\setlength{\parindent}{0mm}


% style definitions
% -----------------
% style categories explanation:
% * \cvnameXXX is used for the name;
% * \cvsectionXXX is used for section names (left column, accompanied by a horizontal rule);
% * \cvtitleXXX is used for job/education titles (right column);
% * \cvdurationXXX is used for job/education durations (left column);
% * \cvheadingXXX is used for headings (left column);
% * \cvmainXXX (and \setmainfont) is used for main text;
% * \cvruleXXX is used for the horizontal rules denoting sections.

% font families
\defaultfontfeatures{Ligatures=TeX} % reportedly a good idea, see https://tex.stackexchange.com/a/37251

\newfontfamily{\cvnamefont}{Roboto Medium}
\newfontfamily{\cvsectionfont}{Roboto Medium}
\newfontfamily{\cvtitlefont}{Roboto Regular}
\newfontfamily{\cvdurationfont}{Roboto Light Italic}
\newfontfamily{\cvheadingfont}{Roboto Regular}
\setmainfont{Roboto Light}

% colours
\definecolor{cvnamecolor}{HTML}{000000}
\definecolor{cvsectioncolor}{HTML}{000000}
\definecolor{cvtitlecolor}{HTML}{000000}
\definecolor{cvdurationcolor}{HTML}{000000}
\definecolor{cvheadingcolor}{HTML}{000000}
\definecolor{cvmaincolor}{HTML}{000000}
\definecolor{cvrulecolor}{HTML}{000000}

\color{cvmaincolor}

% styles
\newcommand{\cvnamestyle}[1]{{\Large\cvnamefont\textcolor{cvnamecolor}{#1}}}
\newcommand{\cvsectionstyle}[1]{{\normalsize\cvsectionfont\textcolor{cvsectioncolor}{#1}}}
\newcommand{\cvtitlestyle}[1]{{\large\cvtitlefont\textcolor{cvtitlecolor}{#1}}}
\newcommand{\cvdurationstyle}[1]{{\small\cvdurationfont\textcolor{cvdurationcolor}{#1}}}
\newcommand{\cvheadingstyle}[1]{{\normalsize\cvheadingfont\textcolor{cvheadingcolor}{#1}}}


% inter-item spacing
% ------------------

% vertical space after personal info and standard CV items
\newlength{\cvafteritemskipamount}
\setlength{\cvafteritemskipamount}{5mm plus 1.25mm minus 1.25mm}

% vertical space after sections
\newlength{\cvaftersectionskipamount}
\setlength{\cvaftersectionskipamount}{2mm plus 0.5mm minus 0.5mm}

% extra vertical space to be used when a section starts with an item with a heading (e.g. in the skills section),
% so that the heading does not follow the section name too closely
\newlength{\cvbetweensectionandheadingextraskipamount}
\setlength{\cvbetweensectionandheadingextraskipamount}{1mm plus 0.25mm minus 0.25mm}


% intra-item spacing
% ------------------

% vertical space after name
\newlength{\cvafternameskipamount}
\setlength{\cvafternameskipamount}{3mm plus 0.75mm minus 0.75mm}

% vertical space after personal info lines
\newlength{\cvafterpersonalinfolineskipamount}
\setlength{\cvafterpersonalinfolineskipamount}{2mm plus 0.5mm minus 0.5mm}

% vertical space after titles
\newlength{\cvaftertitleskipamount}
\setlength{\cvaftertitleskipamount}{1mm plus 0.25mm minus 0.25mm}

% value to be used as parskip in right column of CV items and itemsep in lists (same for both, for consistency)
\newlength{\cvparskip}
\setlength{\cvparskip}{0.5mm plus 0.125mm minus 0.125mm}

% set global list configuration (use parskip as itemsep, and no separation otherwise)
\setlist{parsep=0mm,topsep=0mm,partopsep=0mm,itemsep=\cvparskip}


% CV commands
% -----------

% creates a "personal info" CV item with the given left and right column contents, with appropriate vertical space after
% @param #1 left column content (should be the CV photo)
% @param #2 right column content (should be the name and personal info)
\newcommand{\cvpersonalinfo}[2]{
    % left and right column
    \begin{minipage}[t]{\cvleftcolumnwidth}
        \vspace{0mm} % XXX hack to align to top, see https://tex.stackexchange.com/a/11632
        \raggedleft #1
    \end{minipage}% XXX necessary comment to avoid unwanted space
    \hspace{\cvcolumngapwidth}% XXX necessary comment to avoid unwanted space
    \begin{minipage}[t]{\cvrightcolumnwidth}
        \vspace{0mm} % XXX hack to align to top, see https://tex.stackexchange.com/a/11632
        #2
    \end{minipage}

    % space after
    \vspace{\cvafteritemskipamount}
}

% typesets a name, with appropriate vertical space after
% @param #1 name text
\newcommand{\cvname}[1]{
    % name
    \cvnamestyle{#1}

    % space after
    \vspace{\cvafternameskipamount}
}

% typesets a line of personal info beginning with an icon, with appropriate vertical space after
% @param #1 parameters for the \includegraphics command used to include the icon
% @param #2 icon filename
% @param #3 line text
\newcommand{\cvpersonalinfolinewithicon}[3]{
    % icon, vertically aligned with text (see https://tex.stackexchange.com/a/129463)
    \raisebox{.5\fontcharht\font`E-.5\height}{\includegraphics[#1]{#2}}
    % text
    #3

    % space after
    \vspace{\cvafterpersonalinfolineskipamount}
}

% creates a "section" CV item with the given left column content, a horizontal rule in the right column, and with
% appropriate vertical space after
% @param #1 left column content (should be the section name)
\newcommand{\cvsection}[1]{
    % left and right column
    \begin{minipage}[t]{\cvleftcolumnwidth}
        \raggedleft\cvsectionstyle{#1}
    \end{minipage}% XXX necessary comment to avoid unwanted space
    \hspace{\cvcolumngapwidth}% XXX necessary comment to avoid unwanted space
    \begin{minipage}[t]{\cvrightcolumnwidth}
        \textcolor{cvrulecolor}{\rule{\cvrightcolumnwidth}{0.3mm}}
    \end{minipage}

    % space after
    \vspace{\cvaftersectionskipamount}
}

% creates a standard, multi-purpose CV item with the given left and right column contents, parskip set to cvparskip
% in the right column, and with appropriate vertical space after
% @param #1 left column content
% @param #2 right column content
\newcommand{\cvitem}[2]{
    % left and right column
    \begin{minipage}[t]{\cvleftcolumnwidth}
        \raggedleft #1
    \end{minipage}% XXX necessary comment to avoid unwanted space
    \hspace{\cvcolumngapwidth}% XXX necessary comment to avoid unwanted space
    \begin{minipage}[t]{\cvrightcolumnwidth}
        \setlength{\parskip}{\cvparskip} #2
    \end{minipage}

    % space after
    \vspace{\cvafteritemskipamount}
}

% typesets a title, with appropriate vertical space after
% @param #1 title text
\newcommand{\cvtitle}[1]{
    % title
    \cvtitlestyle{#1}

    % space after
    \vspace{\cvaftertitleskipamount}
    % XXX need to subtract cvparskip here, because it is automatically inserted after the title "paragraph"
    \vspace{-\cvparskip}
}


% header and footer
% -----------------

% set empty header and footer
\pagestyle{empty}



% preamble end/document start
% ===========================

\begin{document}


% personal info
% -------------

\cvpersonalinfo{
    % photo
    \includegraphics[height=36mm]{photo.png}
}{
    % name
    \cvname{Vidhyasagar Babalsure}

    % address
    \cvpersonalinfolinewithicon{height=4mm}{072-location.pdf}{
        Dighi, Pune 411015 
    }

    % phone number
    \cvpersonalinfolinewithicon{height=4mm}{067-phone.pdf}{
        +917709510027
    }

    % email address
    \cvpersonalinfolinewithicon{height=4mm}{070-envelop.pdf}{
        \href{mailto:sagarbabalsure@gmail.com}{sagarbabalsure@gmail.com}
    }

    % LinkedIn account
    \cvpersonalinfolinewithicon{height=4mm}{458-linkedin.pdf}{
       \textit{\href{https://www.linkedin.com/in/vidhyasagar-babalsure-5277b8185/}{vidhyasagar-babalsure-5277b8185}}
        
    }
    
    % Github  account
    \cvpersonalinfolinewithicon{height=4mm}{github.pdf}{
        \textit{\href{https://github.com/sagarbabalsure}{sagarbabalsure}}
    }

    % date of birth
    DOB 12 Feb 2000
}


% work experience
% ---------------

\cvsection{PROJECTS}

% Fake Company 2
\cvitem{
    \cvdurationstyle{July 2020 -- present}
}{
    \cvtitle{Expense Tracker}

    Daily expense tracker 

    \begin{itemize}[leftmargin=*]
        \item The Purpose of this application is to track our daily expenses hassle-free.
        \item This web application is using Django 3 as backend, HTML, CSS, JS as front-end and PostgreSQL as a database.
        \item PROJECT LINK: 
         \textit{\href{https://github.com/sagarbabalsure/ExpenseTracker}{github.com/sagarbabalsure/ExpenseTracker}}
    \end{itemize}
 }

% Fake Company 1
\cvitem{
    \cvdurationstyle{March 2020 -- April 2020}
}{
    \cvtitle{Youtube downloader}

    % Fake Company 1, Fake City

    \begin{itemize}[leftmargin=*]
         \item This is a simple command-line interface for downloading videos and audios from youtube. I have implemented it using Python3.6. For downloading just paste the youtube link and select resolution.
         \item PROJECT LINK: \textit{\href{https://github.com/sagarbabalsure/Youtube_downloader}{github.com/sagarbabalsure/Youtube\_downloader}}
    \end{itemize}
}

% Fake Company 1
\cvitem{
    \cvdurationstyle{Dec 2019 -- Jan 2020}
}{
    \cvtitle{Live Cricket Score}

    % Fake Company 1, Fake City

    \begin{itemize}[leftmargin=*]
         \item Scraped the live cricket score from crickbuzz website. Terminal based app using Python3(beautifulsoup4, requests, html5lib).
         \item PROJECT LINK: \textit{\href{https://github.com/sagarbabalsure/Live-Cricket}{github.com/sagarbabalsure/Live-Cricket}}
    \end{itemize}
}

% Fake Company 1
\cvitem{
    \cvdurationstyle{Jul 2019 -- Sep 2019}
}{
    \cvtitle{Medicart}

    % Fake Company 1, Fake City

    \begin{itemize}[leftmargin=*]
         \item Flask based web application provides an online platform for buying and selling medicines.
        \item Front-end is implemented using HTML, CSS, JS and sqlalchemy used for handling databases.
         \item PROJECT LINK: \textit{\href{https://github.com/sagarbabalsure/Medicart}{github.com/sagarbabalsure/Medicart}}
    \end{itemize}
}

% Fake Company 1
\cvitem{
    \cvdurationstyle{Dec 2018 -- Mar 2019}
}{
    \cvtitle{Tour Management System}

    % Fake Company 1, Fake City

    \begin{itemize}[leftmargin=*]
         \item Terminal based Online Tour and Traveling Booking is a system that gives customers the facility of booking Tours from available packages. Implemented in C++ using OOP concepts. 
    \end{itemize}
}

% Fake Company 1
\cvitem{
    \cvdurationstyle{Jul 2018 -- Sep 2018}
}{
    \cvtitle{Stationary Distribution System}

    % Fake Company 1, Fake City

    \begin{itemize}[leftmargin=*]
         \item This system helps shopkeepers to keep track of products and this software enables users to create or add a new item in the stationary list.
         \item Implemented in C language using various data structures.
         \item PROJECT LINK: 
         \textit{\href{https://github.com/sagarbabalsure/StationaryDistributionSystem}{github.com/sagarbabalsure/StationaryDistributionSystem}}
    \end{itemize}
}


% education
% ---------

\cvsection{EDUCATION}

% master's
\cvitem{
    \cvdurationstyle{2017 -- present}
}{
    \cvtitle{Bachelor of Technology}

    G.H. Raisoni College of Engineering and Management, Pune.

    \begin{itemize}[leftmargin=*]
        \item CGPA - \textbf{9.52/10}
        \item Year of completion - 2021
    \end{itemize}
}

% bachelor's
\cvitem{
    \cvdurationstyle{2016 -- 2017}
}{
    \cvtitle{HSC}

   Modern College of Arts, Science and Commerce Shivajinagar, Pune-411005

    \begin{itemize}[leftmargin=*]
        \item Percentage - \textbf{79.08\%}
    \end{itemize}
}
\bigskip

% skills
% ------

\cvsection{SKILLS}

\vspace{\cvbetweensectionandheadingextraskipamount}

% languages
\cvitem{
    \cvheadingstyle{Languages}
}{
    \item Python, C++, Java, C
    % Fake language -- fake proficiency description
    % \begin{itemize}
    %     \item fake certificate description
    % \end{itemize}
}


\cvitem{
    \cvheadingstyle{Web Development}
}{
    \item Front End -- HTML, CSS, JAVASCRIPT, Jquery
    \item Back End  -- Flask, Django
    \item Database  -- MySQL, SQLite, MongoDB 
    % Fake language -- fake proficiency description
    % \begin{itemize}
    %     \item fake certificate description
    % \end{itemize}
}

\cvitem{
    \cvheadingstyle{Tools}
}{
    \item Git, Github, Jupyter
}
\cvitem{
    \cvheadingstyle{Other}
}{
    \item Data structures and Algorithms, Machine Learning
}

\cvsection{CERTIFICATIONS}

\cvitem{
    \cvdurationstyle{July 2020}
}{
    \cvtitle{Machine learning with python, Coursera}

    % Fake Company 1, Fake City

    \begin{itemize}[leftmargin=*]
         \item I have completed this course with \textbf{100.00\%} grades. Learned about regression, classification and clustering algorithms.
        \item URL: 
        \textit{\href{https://www.coursera.org/account/accomplishments/certificate/T4ZNVF93GQF8}{https://bit.ly/3f6kWQh}}
    \end{itemize}
}

\cvitem{
    \cvdurationstyle{April 2020}
}{
    \cvtitle{Python Data Structures, Coursera}

    % Fake Company 1, Fake City

    \begin{itemize}[leftmargin=*]
         \item I have completed this course with \textbf{100.00\%} grades. This course covers the data structures of
         python programming.
        \item URL:
       \textit{\href{https://www.coursera.org/account/accomplishments/certificate/AKZ5LTN9JV5L}{https://bit.ly/3jB8U5b}}
    \end{itemize}
}


\cvitem{
    \cvdurationstyle{April 2020}
}{
    \cvtitle{Programming for Everybody, Coursera}

    % Fake Company 1, Fake City

    \begin{itemize}[leftmargin=*]
         \item I have completed this Coursera course with \textbf{98.30\%} grades.
        \item URL:
        \textit{\href{https://www.coursera.org/account/accomplishments/certificate/7APPEQDNGXJX}{https://bit.ly/3hDW8Ro}}
        
    \end{itemize}
}


\cvitem{
    \cvdurationstyle{Oct 2019}
}{
    \cvtitle{Django, Udemy}

    % Fake Company 1, Fake City

    \begin{itemize}[leftmargin=*]
         \item I have completed Django ( python-based framework
used for back end development ) course from Udemy.
        \item URL:
        \textit{\href{https://www.udemy.com/certificate/UC-IOQDAJNX/}{https://bit.ly/2WZP8GJ}}
        
    \end{itemize}
}






\cvitem{
    \cvdurationstyle{Jan 2019 -- Mar 2019}
}{
    \cvtitle{Programming in C++, NPTEL}

    % Fake Company 1, Fake City

    \begin{itemize}[leftmargin=*]
         \item I have completed 'PROGRAMMING IN C++' 8 weeks
course on NPTEL with an \textbf{"ELITE"} medal.
        \item URL:
        \textit{\href{https://nptel.ac.in/noc/Ecertificate/?q=noc19-cs10/NPTEL19CS10S21740556191063293.jpg}{https://bit.ly/3332Dcm}}
        
    \end{itemize}
}








% additional info
% ---------------

\cvsection{EXTRA CURRICULUM ACTIVITIES}

\vspace{\cvbetweensectionandheadingextraskipamount}

\cvitem{
    \cvheadingstyle{May 2020}
}{
    Attended online workshop on Basics of machine learning organised by CREATES, IIT Madras and a webinar on Artificial Intelligence organised by KVCH.
}



\cvitem{
    \cvheadingstyle{Feb 2020 and Sep 2019}
}{
   Participated in different colleges competitions like CODEJUNKIE, CODEBUZZ
}

\cvitem{
    \cvheadingstyle{OCT 2019}
}{
  Contributed to Hactoberfest-2019 and won goodies.
}


\cvitem{
    \cvheadingstyle{Mar 2019}
}{
  Attended Blockchain Technology and Application conference organized by TRAI at Pune.
}

\cvitem{
    \cvheadingstyle{Sep 2018}
}{
    Participated in college-level DEBATE COMPETITION which is organized under the ISTE student chapter and got 2nd prize.
}

\cvitem{
    \cvheadingstyle{Mar 2018}
}{
    Presented the paper at ’RGI-National Level Technical/Research Paper Competition ’ on Face Detection and got the 1st prize.
}

\cvsection{PERSONAL DETAILS}

\cvitem{
    \cvheadingstyle{Languages}
}{
    \item English, Hindi, Marathi
}

\end{document}